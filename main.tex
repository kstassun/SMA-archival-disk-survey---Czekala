% \documentclass[twocolumn]{aastex6}
\documentclass{aastex6}

\newcommand{\radmc}{\texttt{RADMC-3D}}
\newcommand{\kms}{ \textrm{km s}^{-1} }
\newcommand{\todo}[1]{ \textcolor{red}{#1}}
\newcommand{\vt}{ {\bm \theta}}
\newcommand{\msun}{M$_\odot$}

\shorttitle{SMA Dynamical Masses}
\shortauthors{Ian Czekala}

\begin{document}

\title{SMA Dynamical Masses}
\author{I.~Czekala\altaffilmark{1}, S.~M.~Andrews\altaffilmark{1}, D.~J.~Wilner\altaffilmark{1}, E.~L.~N.~Jensen\altaffilmark{2}, and K.~G.~Stassun\altaffilmark{3,4}}
\altaffiltext{1}{Harvard-Smithsonian Center for Astrophysics,
			 60 Garden Street, Cambridge, MA 02138; \email{iczekala@cfa.harvard.edu}}
\altaffiltext{2}{Department of Physics and Astronomy, Swarthmore College, 500 College Avenue, Swarthmore, PA 19081}
\altaffiltext{3}{Department of Physics and Astronomy, Vanderbilt University, Nashville, TN 37235}
\altaffiltext{4}{Department of Physics, Fisk University, Nashville, TN 37208}

\begin{abstract}
We present a survey of precision mass measurements for a sample of 19 well known Herbig Ae/Be and T Tauri stars. We use resolved observations with the Submillimeter Array of the Keplerian rotation of CO gas to make a precise, \emph{dynamical} mass measurement of a sample of \emph{single} stars, with 10 sources to a precision of better than 10\%. We report distance-independent quantity $M_\ast/d$ to better precision than 10\%, with distance being the main source of uncertainty, for which \emph{GAIA} will shortly ameliorate.

We use these measurements, coupled with constraints on effective temperature and luminosity, to compare to pre-main sequence model predictions in a distance-independent manner, showing that while the models do well across a variety of spectral types, there are often consistent systematic disagreements (maybe).
\end{abstract}


\section{Introduction}

Dynamical masses for all the SMA data. These sources have long been studied and are important to have accurate measurements.

Other important compilations include Hillenbrand and White, Guilloteau, Simon, Dutrey.

% Paragraph about the ability of the dynamical mass technique to measure the masses of stars.
% Tomographic reconstruction of the velocity field.
% Note the success of our groups measurements for young binaries.
% Ultimate promise is to apply this to a large measurement of single stars.
% The ultimate goal is large programs with ALMA, but there are a number of disks that have been observed with SMA for which we can already do this measurement.
We now have a technique available that delivers accurate masses \citep{rosenfeld12b, czekala15a, czekala16}.

\section{Data and Reduction}

Calibration. MIRIAD software. Visibilities were exported from UVFITS to UVHDF5\footnote{See \url{https://github.com/Astrochem/UVHDF5}} format.

The data for most of the sources in this program have been previously published. For references, see Table~\ref{table:targets}.

\begin{deluxetable*}{lccccccc}
 \tablecaption{\label{table:targets}Targets}
  \tablehead{\colhead{{Source}} & \colhead{R.A.} & \colhead{DEC} & \colhead{SPT} & \colhead{${}^{12}$CO $J$=} & \colhead{$d$ [pc]} & \colhead{Ref}}
 \startdata
LkH$\alpha$ 330 & 03:45:48.28 & +32:24:11.9 & G1--G5 & 3-2  & $315$ & ,1,9 \\ %schlafly14, andrews11a
UX Tau A         & 04:30:04.00 & +18:13:49.4 & K1--K5 & 3-2 & $145$ & ,2,9 \\ %torres10, andrews11a
DM Tau          & 04:33:48.72 & +18:10:09.9 & K7--M2 & 2-1 & $145$ & ,2,10 \\ %torres10, oberg10
AA Tau          & 04:34:55.42 & +24:28:53.2 & K6--M1 & 3-2 & $145$ & ,2,11 \\ %torres10, andrews07
LkCa 15         & 04:39:17.80 & +22:21:03.5 & K3--K7 & 3-2 & $145$ & ,2,11 \\% torres10, andrews11b
GM Aur          & 04:55:10.98 & +30:21:59.5 & K2--K5 & 3-2 & $145$ & ,2,12 \\ %torres10, hughes09
MWC 480         & 04:58:46.26 & +29:50:37.1 & A2--A8 & 2-1 & $145$ & ,2,10 \\ %torres10, oberg10
MWC 758         & 05:30:27.53 & +25:19:57.1 & A5--F0 & 3-2 & $279$ & ,3,13 \\% vanleeuwen07, isella10
CQ Tau          & 05:35:58.46 & +24:44:54.2 & F2--F6 & 2-1 & $113$ & ,3,10 \\%vanleeuwen07, oberg10
TW Hya          & 11:01:51.91 & -34:42:17.0 & K6--M3 & 3-2 & $54$ & ,3,14 \\ %vanleeuwen07, andrews12
SAO 206462      & 15:15:48.44 & -37:09:16.0 &  F6--G0 & 2-1 & $145$ & ,2,15 \\% torres10, lyo11
IM Lup          & 15:56:09.18 & -37:56:06.1 & K7--M2 & 2-1 & $155$ & ,4 \\% lombardi08, none
HD 142527       & 15:56:41.89 & -42:19:23.3 & F2--G0 & & $145$ & ,2 \\%torres10, none
RX J1604.3-2130 & 16:04:21.66 & -21:30:28.4 & K0--K3 & 3-2 & $145$  & ,5,16 \\%dezeeuw99, mathews12
RX J1615.3-3255 & 16:15:20.23 & -32:55:05.1 & K4--K7 & 3-2 & $185$ & ,6,9 \\%makarov07, andrews11a
RX J1633.9-2442 & 16:33:55.61 & -24:42:05.0 & K5--M0 & & $120$ & ,7,17 \\% loinard08, cieza12
AS 209          & 16:49:15.30 & -14:22:08.6 & K3--K6 & 3-2 & $130$ & ,3,18 \\%vanleeuwen07, andrews09
HD 163296       & 17:56:21.29 & -21:57:21.9 & A0--A4 & 2-1 & $119$ & ,3,19 \\% vanleeuwen07, qi11
HD 169142       & 18:24:29.78 & -29:46:49.4 & B8--A2 & 2-1 & $145$ & ,8,20 \\% vanboekel05, raman06
 \enddata
 \tablecomments{Comma-separated references are for spectral type, source distance, and previous publication of data, if one exists. (1) \citet{schlafly14}, (2) \citet{torres10}, (3) \citet{vanleeuwen07}, (4) \citet{lombardi08}, (5) \citet{dezeeuw99}, (6) \citet{makarov07}, (7) \citet{loinard08}, (8) \citet{vanboekel05}, (9) \citet{andrews11a}, (10) \citet{oberg10}, (11) \citet{andrews07}, \citet{andrews11b}, (12) \cite{hughes09}, (13) \citet{isella10}, (14) \citet{andrews12}, (15) \citet{lyo11}, (16) \citet{mathews12}, (17) \citet{cieza12}, (18) \citet{andrews09}, (19) \citet{qi11}, (20) \citet{raman06}.}
\end{deluxetable*}

\section{Methodology}

\subsection{Millimeter Interferometry}

Interferometers sample the Fourier transform of the sky brightness of an astrophysical source and record these measurements as complex-valued data visibilities. By forward modeling these visibilities, we can use a well-motivated likelihood function, coupled with physical prior distributions, to evaluate the posterior probability of our model parameters as well as understand the uncertainties in our inference.

The first step in this modeling procedure is to create a model of the disk molecular line emission in the sky plane. This involves parametrically defining a 2D-axisymmetric model of the gas disk structure and velocity field in cylindrical coordinates. Since we fit the visibility data after they have been dust continuum-subtracted, we do not consider dust emission in this model. The gas is assumed to orbit in the disk in circular Keplerian orbits, with only an azimuthal component of velocity that is a function of the central stellar mass, $M_\ast$
\begin{equation}
	v_\phi(r) = \sqrt{\frac{G M_\ast}{r}}
\end{equation}
The temperature of the gas is defined parametrically by a power-law profile with a normalization at 10 AU $T_{10}$ and gradient $q$, and the gas is assumed to be vertically isothermal
\begin{equation}
	T(r) = T_{10} \left ( \frac{r}{\textrm{10 AU}}\right)^{-q}
\end{equation}
We assume that the standard \citet{lynden-bell74} similarity solution describes the surface density profile,
\begin{equation}
\Sigma(r) = \Sigma_c \left (\frac{r}{r_c} \right)^{- \gamma} \exp \left[ - \left(\frac{r}{r_c} \right)^{2 - \gamma} \right]
\end{equation}
 where the critical radius $r_c$ denotes the transition from a power law to exponential taper and the pre-factor $\Sigma_c$ normalizes the profile. The total disk mass ($M_\textrm{disk}$) can be found by integrating $\Sigma(r)$ over the surface area of the disk. We proceed with fixed $\gamma = 1$ and term this the ``standard'' surface density profile for later discussion. Some of the disks in our sample exhibit large gas density depletions in a central cavity near the star. To more accurately model these disks, we alternatively employ a ``cavity'' model which has two additional parameters $r_\mathrm{cav}$ and $\gamma_c$
\begin{equation}
\Sigma(r) = \Sigma_c \left (\frac{r}{r_c} \right)^{- \gamma} \exp \left [- \left ( \frac{r_\mathrm{cav}}{r} \right)^{\gamma_c} \right ]  \exp \left[ - \left(\frac{r}{r_c} \right)^{2 - \gamma} \right]
\end{equation}
which parameterize an inner cavity with variable size and steepness of decay.

% We elect to use a model with a smooth exponential transition rather than more conventional cavity models (such as a simple depletion factor inside some radius) because of the finite grid used to do the radiative transfer. It is necessary to ``resolve'' the transition in the radiative transfer to reduce the entrance of numerical artifacts, and this is impossible to do with a model with a sharp transition.
The disk is assumed to be in hydrostatic equilibrium, yielding a vertical gas density structure given by
\begin{equation}
\rho(r, z) = \frac{\Sigma(r)}{\sqrt{2 \pi} H} \exp \left [- \frac{z^2}{2 H^2} \right]
\end{equation}
where the scale height $H$ is given by
\begin{equation}
	H(r) = \frac{c_s}{\Omega} = \left (\frac{k_\mathrm{B} T}{\mu m_\mathrm{H}} \cdot \frac{r^3}{G M_\ast} \right)
\end{equation}
where $c_s$ is the sound speed, $\Omega$ is the angular velocity, $k_\mathrm{B}$ is the Boltzmann constant, $\mu$ is the gas mean molecular weight, and $m_\mathrm{H}$ is the mass of a Hydrogen atom.

This structure is then ray-traced using the radiative transfer program \texttt{RADMC-3D} in order to produce channel maps. At this stage, the inclination ($i$) and position angle ($\varphi$) of the disk are taken into account. Inclination is defined using the angle between the disk angular momentum vector to the observer line of sight: 0\degr is a face-on disk with the vector pointed towards the observer, 90\degr is an edge-on disk, and 180\degr is a face-on disk with the vector pointed away from the observer. Similarly, position angle is defined using the projection of the disk angular momentum vector on the plane of the sky, in degrees running East of North (counter clockwise on the sky). Using an assumed fixed distance to the source, we then scale these channel maps to the appropriate size and intensity. Systematic uncertainty in the distance to the source translates linearly into uncertainty in the stellar mass.

Next, the channel maps are Fourier transformed using \texttt{FFTW} \citep{fftw} and then sampled at the same $u, v$ coordinates corresponding to the antenna baselines of the observations. This process is carried out in a band-limited manner that respects the gridding convolution functions \citep{schwab84}. At this point, the visibilities are also multiplied by a phase-center offset corresponding to a position-offset in the image plane, with parameters $\mu_\alpha$ and $\mu_\delta$.

This model describes a kinematic fingerprint, we can infer parameters of the disk and precisely constrain the central stellar mass. \citet{rosenfeld13a} have shown that terms like vertical geometry, disk-self gravity, and radial pressure gradient amount to small contributions on the mass inference.


% We use the dynamical mass technique described in \citet{czekala15a}.





\subsection{Priors}

We enforce a geometrical prior on the inclination of the disk, $p(i_d) =  \sin(i_d)/2$, designed to account for the fact that a disk is more likely to be observed edge-on than face-on. For the cavity model, we set a prior that $r_\textrm{cav} \leq r_c$ and $\gamma_c \geq 0$.




Explore this posterior with MCMC on a cluster. After running multiple chains to asses convergence, burn-in is removed and posteriors are calculated. A full example of the 12-dimensional posterior is shown in Figure~\ref{fig:posterior}.


\footnote{We have released this dynamical mass code as a publicly available, open-source (under an MIT liscence) package written in the \texttt{Julia} language, available at \url{https://github.com/iancze/DiskJockey}}

\subsection{Photospheric Stellar Fitting}

\section{Results}

We present our results in Table~\ref{table:masses}.

We present a table of the model parameters shared by each model, and list the evaluations done by each model in the table. As you can see, there are minimal changes in stellar mass.

Note that because the disk structures are derived from optically thick transitions, we caution about the interpretability of the disk structure itself, particularly inference about the total disk mass. Previous fits using these transitions have been shown to yield accurate stellar masses, and $M_\ast$ in generally is rather insensitive to small inaccuracies in disk structure. In essentially, these disk structure parameters are carted around as nuisance parameters to fit the disk.

% I think an interesting figure for this section would be to plot a representative set of PMS tracks (Dartmouth), then plot the existing disk-based sources in transparent blue, plot the existing eclipsing binary sources in transparent red, then plot our updated disk based measurements in a bold blue. Below, we will have a histogram showing where these have been filled in.

% Stars that are in Tycho-2 catalog, and will presumably have astrometric solutions from the first GAIA release, on September 14th.
% LkHa 330
% UX Tau A
% LkCa 15
% GM Aur
% MWC 480
% MWC 758
% CQ Tau
% TW Hya
% SAO 206462
% HD 142527
% AS 209
% HD 163296
% HD 169142
% that's 13 / 19 sources, which is excellent.

\begin{deluxetable*}{lcccccccccccc}
 \tablecaption{\label{table:masses}Inferred System Properties}
  \tablehead{\colhead{{Source}} & \colhead{$M_\ast$} & \colhead{$i_d$} & \colhead{$r_c$} & \colhead{$\log M_\textrm{CO}$} & \colhead{$T_{10}$} & \colhead{$q$} & \colhead{$\xi$} & \colhead{P.A.} & \colhead{$r_\mathrm{cav}$} & \colhead{$\gamma_\mathrm{cav}$} \\
	& \colhead{[$M_\odot$]} & \colhead{[${}^\circ$]} & \colhead{[AU]} & \colhead{[$M_\odot$]} & \colhead{[K]} & & \colhead{ [$\kms$]} & \colhead{[${}^\circ$]} & \colhead{[AU]} & }
 \startdata
 LkH$\alpha$ 330 & \\
 UX Tau A & \\
 DM Tau & \\
 AA Tau & \\
 LkCa 15 & $1.11_{-0.18}^{+0.09}$ & $45.7_{-2.3}^{+1.1}$ & $470_{-131}^{+105}$ & $-3.9_{-0.2}^{+0.2}$ & $94_{-14}^{+6}$ & $0.57_{-0.04}^{+0.02}$ & $0.48_{-0.05}^{+0.02}$ & $153.2_{-1.0}^{+1.0}$  \\
 GM Aur & \\
 MWC 480 & $2.14_{-0.51}^{+0.22}$ & $35.2_{-1.4}^{+0.4}$ & $136_{-21}^{+45}$ & $-3.7_{-0.1}^{+0.1}$ & $325_{-39}^{+29}$ & $0.75_{-0.01}^{+0.02}$ & $0.29_{-0.01}^{+0.01}$ & $58.4_{-0.2}^{+0.4}$ \\
 MWC 758 & \\
 CQ Tau & \\
 TW Hya & \\
 SAO 206462 & \\
 IM Lup & \\
 HD 142527 & \\
 RX J1604 & \\
 RX J1615 & \\
 RX J1633 & \\
 AS 209 & \\
 HD 163296 & \\
 HD 169142 & \\
% LkH$\alpha$ 330 & \tablenotemark{a}  \\
% UX Tau A         &\\
% DM Tau          & \\
% AA Tau          & \\
% LkCa 15         & \\
% GM Aur          & \\
% MWC 480         & \\
% MWC 758         & \\
% CQ Tau          & \\
% TW Hya          & \\
% SAO 206462      & \\
% IM Lup          & \\
% HD 142527       & \\
% RX J1604 				& \\
% RX J1615 				& \\
% RX J1633 				& \\
% AS 209          & \\
% HD 163296       & \\
% HD 169142       & \\
 \enddata
 \tablenotetext{a}{From lack of sufficient sensitivity in the channels at the systemic velocity or any external constraints, it is unknown whether the absolute orientation of the disk inclination vector is pointed towards ($0\degr \leq i_d \leq 90\degr$) or away from ($90\degr \leq i_d \leq 180\degr$) the observer.}
 \tablecomments{Full posterior and likelihood samples are provided for download \todo{at figshare HERE.}}
\end{deluxetable*}

\begin{figure*}[htb]
\begin{center}
  \includegraphics{triangle.pdf}
  \figcaption{
  A giant, 11 parameter figure showing the full parameter space. Explaining the various parameters and how they are connected.
  \label{fig:posterior}}
  \end{center}
\end{figure*}

\begin{figure*}[htb]
\begin{center}
\includegraphics[draft, width=2.3in, height=2.15in]{LkHa330_posterior.pdf}
\includegraphics{UXTauA_posterior.pdf}
\includegraphics{DMTau_posterior.pdf}
\includegraphics[draft, width=2.3in, height=2.15in]{AATau_posterior.pdf}
\includegraphics{LkCa15_posterior.pdf}
\includegraphics{GMAur_posterior.pdf}
\includegraphics{MWC480_posterior.pdf}
\includegraphics{MWC758_posterior.pdf}
\includegraphics[draft, width=2.3in, height=2.15in]{CQTau_posterior.pdf}
\figcaption{All of the tiny posteriors.
\label{fig:posteriors_one}
}
\end{center}
\end{figure*}

\begin{figure*}[htb]
\begin{center}
\includegraphics[draft, width=2.3in, height=2.15in]{TWHya_posterior.pdf}
\includegraphics[draft, width=2.3in, height=2.15in]{SAO206462_posterior.pdf}
\includegraphics{IMLup_posterior.pdf}
\includegraphics[draft, width=2.3in, height=2.15in]{HD142527_posterior.pdf}
\includegraphics[draft, width=2.3in, height=2.15in]{RXJ1604_posterior.pdf}
\includegraphics{RXJ1615_posterior.pdf}
\includegraphics[draft, width=2.3in, height=2.15in]{RXJ1633_posterior.pdf}
\includegraphics[draft, width=2.3in, height=2.15in]{AS209_posterior.pdf}
\includegraphics[draft, width=2.3in, height=2.15in]{HD163296_posterior.pdf}
\figcaption{All of the tiny posteriors.
\label{fig:posteriors_two}
}
\end{center}
\end{figure*}

\begin{figure}[htb]
\begin{center}
\includegraphics{HD169142_posterior.pdf}
\figcaption{All of the tiny posteriors.
\label{fig:posteriors_three}
}
\end{center}
\end{figure}

\subsection{Comparison to pre-main sequence evolutionary models}

We use literature values for the estimate of spectral type, effective temperature, and luminosity to place these sources on the pre-main sequence Hertzsprung Russel sp? (HR) diagram. We use a selection of pre-main sequence models from the X, Y, and Z. And we color-code the discrepancy between our measurement and the predicted mass from the photospheric properties.

\begin{figure*}[htb]
\begin{center}
  \includegraphics[draft,width=\textwidth,height=0.5\textheight]{PMS.pdf}
  \figcaption{
  PMS comparison HR diagram, showing probability of agreement as color coded circle.
  \label{fig:PMS}}
  \end{center}
\end{figure*}


\section{Conclusions}

We present the largest compilation of disk-based dynamical mass measurements to date.

We find that the pre-main sequence models (do not) work for X, Y, and Z.

\acknowledgments
IC is supported by the Smithsonian Institution. The authors would like to acknowledge Adam Kraus for helpful discussions regarding the distance to LkH$\alpha$~330; Jane Huang for helpful conversations about the cloud contamination of AS~209. \todo{SA acknowledges XX}. \todo{SMA observers of this data?}  Figures \todo{XX} were generated with \texttt{triangle.py} \citep{foreman-mackey14}. This research made extensive use of the Julia programming language \citep{julia12} and Astropy \citep{astropy13}.

\software{Julia, \citep{julia12}}

\bibliographystyle{yahapj.bst}
\bibliography{SMA.bib}

\appendix

\section{Remarks on individual sources}

\subsection{LkH$\alpha$ 330}
\begin{figure*}[htb]
\begin{center}
  \includegraphics{LkHa330.pdf}
  \figcaption{
  Model, data, and residual channel maps for LkHa~330. Contours are every 3 sigma. Negative residuals are denoted by dashed contours. Color bar labels intensity in multiples of the RMS as well as Jy/beam.
  \label{fig:LkHa330}}
  \end{center}
\end{figure*}

\citet{isella13} find azimuthal asymmetry in the dust ring. Dynamical interactions by unseen low mass planets.
\citep{andrews11a} find the following parameters:
\citep{pontoppidan11} use CRIRES. Shows narrow, blended peak characteristic of a face-on disk at 9 kms LSR. Assuming M=2.5, they derive i = 12, however these are clearly degenerate. For M = 1.6, 17 degrees. If our results are funky, this might be nice to combine as well.

PA=218. CO v = 1 - 0.

% Andrews et al. 2011 determine the following parameters from dust + photometry
% SPT = G3.
% dpc = 250 pre
% M = 2.5 sun.
% i = 35 deg.
% r > 130 AU.
% Dust cavity of 40 AU.

% Strangely, gas modeling does NOT support this mass and inclination combination.

\subsection{UX Tau A}

Although it looks good in the channel maps, this fit is clearly wrong. The disk does not contain $10^5 M_odot$ of material.

\begin{figure*}[htb]
\begin{center}
  \includegraphics{UXTauA.pdf}
  \figcaption{
  Same as above, for UX~Tau~A.
  \label{fig:UXTauA}}
  \end{center}
\end{figure*}

\citep{espaillat07}.

\citet{tanii12} say west side is nearest, so we use negative inclination.

% i = 46 +/- 2
% west side nearest
% no evidence of a gap up to 20 AU.

\subsection{DM Tau}
Previous measurements.

Dartois 03 paper CO channel maps (at higher spatial and velocity resolution) suggest that the inclination is > 90 degrees. Subsequent analysis (Teague, Loomis) assume ~35 degrees inclination and make no attempt to disambiguate absolute orientation of inclination.

\begin{figure*}[htb]
\begin{center}
  \includegraphics{DMTau.pdf}
  \figcaption{
  Same as above, for DM~Tau.
  \label{fig:DMTau}}
  \end{center}
\end{figure*}

\subsection{AA Tau}

\begin{figure*}[htb]
\begin{center}
  \includegraphics[draft, width=0.95\textwidth, height=5in]{AATau.pdf}
  \figcaption{
  Same as above, for AA~Tau
  \label{fig:AATau}}
  \end{center}
\end{figure*}

\subsection{LkCa 15}
Previous measurements. Transition disk.

\citep{vandermarel15} notes PA = 60 deg, i = 55 deg,  vsrc=6.1.

What do we say about the presence of a gap?

\begin{figure*}[htb]
\begin{center}
  \includegraphics{LkCa15.pdf}
  \figcaption{
  Same as above, for LkCa~15.
  \label{fig:LkCa15}}
  \end{center}
\end{figure*}

\subsection{GM Aur}
Previous measurements.

\begin{figure*}[htb]
\begin{center}
  \includegraphics{GMAur.pdf}
  \figcaption{
  Same as above, for GM~Aur.
  \label{fig:GMAur}}
  \end{center}
\end{figure*}


\subsection{MWC 480}
\begin{figure*}[htb]
\begin{center}
  \includegraphics{MWC480.pdf}
  \figcaption{
  Same as above, for MWC~480.
  \label{fig:MWC480}}
  \end{center}
\end{figure*}

\subsection{MWC 758}

\begin{figure*}[htb]
\begin{center}
  \includegraphics{MWC758.pdf}
  \figcaption{
  Same as above, for MWC~758.
  \label{fig:MWC758}}
  \end{center}
\end{figure*}

\subsection{CQ Tau}
% Parallax from van Leeuwen
% 8.85 [1.80]

\begin{figure*}[htb]
\begin{center}
  \includegraphics[draft, width=0.95\textwidth, height=5in]{CQTau.pdf}
  \figcaption{
  Same as above, for CQ~Tau.
  \label{fig:CQTau}}
  \end{center}
\end{figure*}

\subsection{TW Hya}

\begin{figure*}[htb]
\begin{center}
  \includegraphics{TWHya.pdf}
  \figcaption{
  Same as above, for TW~Hya.
  \label{fig:TWHya}}
  \end{center}
\end{figure*}

\subsection{SAO 206462}

\begin{figure*}[htb]
\begin{center}
  \includegraphics[draft, width=0.95\textwidth, height=5in]{SAO206462.pdf}
  \figcaption{
  Same as above, for SAO~206462.
  \label{fig:SAO206462}}
  \end{center}
\end{figure*}

\subsection{IM Lup}

Galli and Bertout quote the SPM4 (Girard et al 2011) catalog for a parallax of 5.3 +/- 1.4 milliarcseconds.

I convert this to 188.7 +/-
149.3 to 256.4

- 40pc + 67 pc

\begin{figure*}[htb]
\begin{center}
  \includegraphics{IMLup.pdf}
  \figcaption{
  Same as above, for IM~Lup.
  \label{fig:IMLup}}
  \end{center}
\end{figure*}

\subsection{HD 142527}

% To watch out for? https://public.nrao.edu/news/pressreleases/binary-star-disk

% M dwarf star is in the transition disk: \url{http://arxiv.org/abs/1511.09390}

% Model the Hercshel data. http://arxiv.org/abs/1606.07266

\begin{figure*}[htb]
\begin{center}
  \includegraphics[draft, width=0.95\textwidth, height=5in]{HD142527.pdf}
  \figcaption{
  Same as above, for HD~142527.
  \label{fig:HD142527}}
  \end{center}
\end{figure*}

\subsection{RX J1604.3-2130}

\begin{figure*}[htb]
\begin{center}
  \includegraphics{RXJ1604.pdf}
  \figcaption{
  Same as above, for RX J1604.3-2130.
  \label{fig:RXJ1604}}
  \end{center}
\end{figure*}

\citep{vandermarel15} notes that this is viewed almost face-on (Mathews 12, Zhang 14). PA = 80, i = 10, vsrc = 4.7 km/s.

Dahm 08, Mathews 12, Carpenter 14

$M = 1.0 M_\odot$.
Spt = K2
d = 145 pc

Dust shadowing by an inner disk. A ``double drop'' going on at two radii.

Gas density drop at 30 AU.

% VLT/Sphere Also see a 30 AU cavity. http://arxiv.org/abs/1606.07087

\subsection{RX J1615.3-3255}
\begin{figure*}[htb]
\begin{center}
  \includegraphics{RXJ1615.pdf}
  \figcaption{
  Same as above, for RX J1615.3-3255.
  \label{fig:RXJ1615}}
  \end{center}
\end{figure*}

\citep{vandermarel15} derives PA = 153, i = 45, vsrc = 4.6 km/s. Says no dust cavity visible. Andrews 11 has cavity of 30 AU.

Large range of permissible CO densities inside gap before CO makes optically thick-thin transition.

\citep{andrews11}
Wichmann et al 1997

Spt = K5

$M = 1.1 M_\odot$.
d = 185 pc.

\subsection{RX J1633.9-2442}

\begin{figure*}[htb]
\begin{center}
  \includegraphics[draft, width=0.95\textwidth, height=5in]{RXJ1633.pdf}
  \figcaption{
  Same as above, for RX J1633.9-2442.
  \label{fig:RXJ1633}}
  \end{center}
\end{figure*}

\subsection{AS 209}

% van Leeuwen 07 parallax
% 7.63 [2.91]

% 95
% 131
% 212

\begin{figure*}[htb]
\begin{center}
  \includegraphics{AS209.pdf}
  \figcaption{
  Same as above, for AS~209.
  \label{fig:AS209}}
  \end{center}
\end{figure*}

Mask central channels due to cloud contamination.

\subsection{HD 163296}

\begin{figure*}[htb]
\begin{center}
  \includegraphics[draft, width=0.95\textwidth, height=5in]{HD163296.pdf}
  \figcaption{
  Same as above, for HD~163296.
  \label{fig:HD163296}}
  \end{center}
\end{figure*}

\subsection{HD 169142}

\begin{figure*}[htb]
\begin{center}
  \includegraphics{HD169142.pdf}
  \figcaption{
  Same as above, for HD~169142.
  \label{fig:HD169142}}
  \end{center}
\end{figure*}

Fit with both standard and cavity models.

Appears to be a definite inner cavity in the gas.

\end{document}
